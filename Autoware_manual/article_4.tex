%%%%%%%%%%%%%%%%%%%%%%%%%%%%%%%%%%%%%%%%%
% Arsclassica Article
% LaTeX Template
% Version 1.1 (1/8/17)
%
% This template has been downloaded from:
% http://www.LaTeXTemplates.com
%
% Original author:
% Lorenzo Pantieri (http://www.lorenzopantieri.net) with extensive modifications by:
% Vel (vel@latextemplates.com)
%
% License:
% CC BY-NC-SA 3.0 (http://creativecommons.org/licenses/by-nc-sa/3.0/)
%
%%%%%%%%%%%%%%%%%%%%%%%%%%%%%%%%%%%%%%%%%

%----------------------------------------------------------------------------------------
%	PACKAGES AND OTHER DOCUMENT CONFIGURATIONS
%----------------------------------------------------------------------------------------

\documentclass[
10pt, % Main document font size
a4paper, % Paper type, use 'letterpaper' for US Letter paper
oneside, % One page layout (no page indentation)
%twoside, % Two page layout (page indentation for binding and different headers)
%headinclude,footinclude, % Extra spacing for the header and footer
BCOR5mm, % Binding correction
]{scrartcl}


\begin{document}

%----------------------------------------------------------------------------------------
%	HEADERS
%----------------------------------------------------------------------------------------




%----------------------------------------------------------------------------------------

\newpage % Start the article content on the second page, remove this if you have a longer abstract that goes onto the second page

%----------------------------------------------------------------------------------------
%	INTRODUCTION
%----------------------------------------------------------------------------------------

\section*{Main Autoware packages }
\subsection*{Localization}
\begin{itemize}

\item \textbf{ndt\_ localizer }

The position of the vehicle can be located by scan matching based on the NDT (Normal Distribution Transform) algorithm with the 3-D map of PCD (Point Cloud Data) format and LIDAR data. The position error of localization is around 10cm. 

\item \textbf{orb\_ localizer }

This is a localizer node that uses maps generated from built-in mapper. In other words, this package is a SLAM localizer. 


\item \textbf{gnss\_ localizer}

Global Navigation Satellite system 

\end{itemize}
 

\subsection*{Detection}
\begin{itemize}
\item \textbf{road\_ wizard }

\item \textbf{cv\_ tracker }

\item \textbf{laidar\_ tracker }

\end{itemize}
\subsection*{Mission}
\begin{itemize}
\item  	\textbf{freespace\_ planner}

\item \textbf{lane\_ planner }

\item \textbf{way\_ planner}
\end{itemize}


\subsection*{Motion Planning} 
\begin{itemize}
 	\item \textbf{astar\_ planner 	}
	\item \textbf{dp\_ planner }	
	\item \textbf{ff\_ waypoint\_ follower }	
	
	\item \textbf{lattice\_ planner}
	
	Lattice based motion planning 	
	\item \textbf{op\_ simulator }	
	\item \textbf{op\_ simulator\_ perception }	
	\item \textbf{waypoint\_ follower} 
	\item \textbf{waypoint\_ maker}
	
	There are 3 types of csv format of route file handled by waypoint\_ maker:
	\begin{itemize}
	\item ver1: It consists of x, y, z and velocity (the first line does not have velocity)
	
	Example:

    3699.6206,-99426.6719,85.8506
    
    3700.6453,-99426.6562,85.8224,3.1646
    
    3701.7373,-99426.6250,85.8017,4.4036
    
    3702.7729,-99426.6094,85.7969,4.7972
    
    3703.9048,-99426.6094,85.7766,4.7954
    
    3704.9192,-99426.5938,85.7504,4.5168
    
    3705.9497,-99426.6094,85.7181,3.6313
    
    3706.9897,-99426.5859,85.6877,4.0757
    
    3708.0266,-99426.4453,85.6608,4.9097
    
    ... 
    
    \item ver 2: It consists of x, y, z, yaw and velocity (the first line does not have velocity)
    
    Example: 
    
    3804.5657,-99443.0156,85.6206,3.1251
    
3803.5195,-99443.0078,85.6004,3.1258,4.8800

3802.3425,-99442.9766,85.5950,3.1279,7.2200

3801.2092,-99442.9844,85.5920,3.1293,8.8600

3800.1633,-99442.9688,85.5619,3.1308,10.6000

3798.9702,-99442.9609,85.5814,3.1326,10.5200

3796.5706,-99442.9375,85.6056,3.1359,10.2200

3795.3232,-99442.9453,85.6082,3.1357,11.0900

3794.0771,-99442.9375,85.6148,3.1367,11.2300

... 
    
    
 \item  x,y,z,yaw,velocity,change_flag
 
 Example:
 
3742.216,-99411.311,85.728,3.141593,0,0

3741.725,-99411.311,85.728,3.141593,10,0

3740.725,-99411.311,85.733,3.141593,10,0

3739.725,-99411.311,85.723,3.141593,10,0

3738.725,-99411.311,85.719,3.141593,10,0

3737.725,-99411.311,85.695,3.141593,10,0

3736.725,-99411.311,85.667,3.141593,10,0

3735.725,-99411.311,85.654,3.141593,10,0

... 
	\end{itemize}
	
\end{itemize}



 




%----------------------------------------------------------------------------------------
%	BIBLIOGRAPHY
%----------------------------------------------------------------------------------------

\renewcommand{\refname}{\spacedlowsmallcaps{References}} % For modifying the bibliography heading

\bibliographystyle{unsrt}

\bibliography{sample.bib} % The file containing the bibliography

%----------------------------------------------------------------------------------------

\end{document}